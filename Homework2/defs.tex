%\newcommand{\proof}{\bf Proof: \rm }%\nr}
\newtheorem{theorem}{Theorem}[section]
\newtheorem{example}{Example}
\newtheorem{lemma}{Lemma}[section]
\newtheorem{corollary}[theorem]{Corollary}
\newtheorem{assumption}{Assumption}
\newtheorem{definition}{Definition}
\newtheorem{property}{Property}
\newtheorem{remark}{Remark}[enumi]

\newcommand{\ones}{\mathbf 1}
\newcommand{\integers}{{\mbox{\bf Z}}}
\newcommand{\symm}{{\mbox{\bf S}}}  % symmetric matrices
\newcommand{\reals}{\mathbb{R}}

\newcommand{\nullspace}{{\mathcal N}}
\newcommand{\range}{{\mathcal R}}
\newcommand{\diag}{\mathop{\bf diag}}
\newcommand{\prox}{\mathbf{prox}}

\newcommand{\Prob}{\mathop{\bf Prob}}
\newcommand{\E}{\mathop{\bf E}}
\newcommand{\Var}{\mathop{\bf Var}}
\DeclareMathOperator*{\argmin}{\arg\!\min}
%\newcommand{\argmin}{\mathop{\rm argmin}}
\newcommand{\argmax}{\mathop{\rm argmax}}
\newcommand{\dom}{\mathop{\bf dom}} % domain
\newcommand{\intr}{\mathop{\bf int}}
\newcommand{\sign}{\mathop{\bf sign}}

\newcommand{\cf}{{\it cf.}}
\newcommand{\eg}{{\it e.g.}}
\newcommand{\ie}{{\it i.e.}}
\newcommand{\etc}{{\it etc.}}
%
%\newcommand{\breg}[2]{D_{\omega}(#1, #2)}
%\newcommand{\bregEuc}[2]{D_{\frac{1}{2}\| \|_2^2}(#1, #2)}
\newcommand{\IndState}[1]{\State\hspace\algorithmicindent{#1}}

\providecommand{\inner}[2]{\langle {#1} , {#2} \rangle}
\providecommand{\norm}[1]{{\left\lVert#1\right\rVert}}	
\providecommand{\abs}[1]{\left | #1 | \right}
\newcommand{\sumn}{\sum_{i = 1}^{n}}

\newcommand{\threepartdef}[6]
{
	\left\{
		\begin{array}{lll}
			#1 & \mbox{if } #2 \\
			#3 & \mbox{if } #4 \\
			#5 & \mbox{if } #6
		\end{array}
	\right.
}

\newcommand{\twopartdef}[4]
{
	\left\{
		\begin{array}{lll}
			#1 & \mbox{if } #2 \\
			#3 & \mbox{ } #4 
		\end{array}
	\right.
}

\newcommand\numberthis{\addtocounter{equation}{1}\tag{\theequation}}

\renewcommand{\qedsymbol}{$\blacksquare$}

\newcommand{\set}[1]{\{#1\}}
\newcommand{\D}{\mathcal{D}}
\newcommand{\Ord}{\mathcal{O}}
\newcommand{\X}{\mathcal{X}}
\newcommand{\R}{{\mathbb R}}

\def\RR{\mathcal{R}}

\newcommand{\derives}{\stackrel{*}{\Rightarrow}}
\newcommand{\Hyp}{{\mathcal H}}   % hypothesis class
\newcommand{\CHyp}[1]{\mathcal{H}_{#1}^{\checkmark}} %correct hyp
\newcommand{\Rad}[1]{\RR_{#1} (F)}  % Rademacher complexity
\newcommand{\err}{{\rm{err}}}

\newcommand{\errh}[1]{{{\rm{err}}_{#1}(h)}}
\newcommand{\MaxGapS}{{\rm {MaxGap}}(S)}  
\newcommand{\MaxGap}{{\rm{MaxGap}} }
\newcommand{\AlgLoss}[1]{\textit{l}_{#1}{\rm{(ALG)}}}
\newcommand{\HypLoss}[1]{\textit{l}_{#1}{\rm{(h)}}}

\newcommand{\bq}{\textbf{q}}
\newcommand{\bp}{\textbf{p}}

\newcommand{\ind}{\mathbbm{1}}
\newcommand{\teps}{\tilde{\varepsilon}}

%alg braces:
\newcommand{\tikzmark}[1]{\tikz[overlay,remember picture] \node (#1) {};}
\newcommand\encircle[1]{%
  \tikz[baseline=(X.base)] 
    \node (X) [draw, shape=circle, inner sep=0] {\strut #1};}
    


\newenvironment{pfsketch}{%
  \renewcommand{\proofname}{Proof Sketch}\proof}{\endproof}